%
% HLMBook - V.Valtchev    26/05/98
%
% Updated - Gareth Moore  15/01/02
%

\newpage
\mysect{LSubset}{LSubset}

\mysubsect{Function}{LSubset-Function}

\index{lsubset@\htool{LSubset}|(}
This program will resolve a word map against a class map and produce a
new word map which contains the class-mapped words. The tool is typically
used to generated a vocabulary-specific $n$-gram word map which is then 
supplied to \htool{LBuild} to build the actual language models.

All class symbols present in the class map will be added to the output
map. The \texttt{-a} option can be used to set the maximum number of
new class symbols in the final word map. Note that the word-class map
resolution procedure is identical to the the one used in \htool{LSubset}
when filtering $n$-gram files.

\mysubsect{Use}{LSubset-Use}

\htool{LSubset} is invoked by typing the command line
\begin{verbatim}
   LSubset [options] inMapFile classMap outMapFile
\end{verbatim}
The word map and class map are loaded, word-class mappings performed and 
a new map is saved to \texttt{outMapFile}. The output map's name will be
set to 
\begin{verbatim}
Name = inMapName%%classMapName
\end{verbatim}

The allowable options to \htool{LSubset} are as follows

\begin{optlist}

  \ttitem{-a n}  Set the maximum number of new classes that can be added 
 	to the output map (default 1000).

\end{optlist}
\stdopts{LSubset}

\mysubsect{Tracing}{LSubset-Tracing}

\htool{LSubset} does not provide any trace options. However, trace 
information is available from the underlying library modules 
\htool{LWMap} and \htool{LCMap} by seeting the appropriate trace
configuration parameters.

\index{lsubset@\htool{LSubset}|)}






