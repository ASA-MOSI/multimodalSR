%/* ----------------------------------------------------------- */
%/*                                                             */
%/*                          ___                                */
%/*                       |_| | |_/   SPEECH                    */
%/*                       | | | | \   RECOGNITION               */
%/*                       =========   SOFTWARE                  */ 
%/*                                                             */
%/*                                                             */
%/* ----------------------------------------------------------- */
%/* developed at:                                               */
%/*                                                             */
%/*      Speech Vision and Robotics group                       */
%/*      Cambridge University Engineering Department            */
%/*      http://svr-www.eng.cam.ac.uk/                          */
%/*                                                             */
%/*      Entropic Cambridge Research Laboratory                 */
%/*      (now part of Microsoft)                                */
%/*                                                             */
%/* ----------------------------------------------------------- */
%/*         Copyright: Microsoft Corporation                    */
%/*          1995-2000 Redmond, Washington USA                  */
%/*                    http://www.microsoft.com                 */
%/*                                                             */
%/*              2001  Cambridge University                     */
%/*                    Engineering Department                   */
%/*                                                             */
%/*   Use of this software is governed by a License Agreement   */
%/*    ** See the file License for the Conditions of Use  **    */
%/*    **     This banner notice must not be removed      **    */
%/*                                                             */
%/* ----------------------------------------------------------- */
%
% HTKBook - Julian Odell and Steve Young 15/11/95
%
% This chapter added to by Gareth Moore 17/01/02

\mychap{Error and Warning Codes}{errors}
\index{errors!full listing}\index{warning codes!full listing}

When a problem occurs in any \HTK\ tool, either
error\index{error message!format}
or warning messages\index{warning message!format} are printed.  

If a warning occurs then a message is sent to standard output and
execution continues. The format of this warning message is as follows:
\begin{verbatim}
    WARNING [-nnnn] Function: 'Brief Explanation' in HTool
\end{verbatim}

The message consists of four parts.  On the first line is the tool
name and the error number\index{error number!structure of}.  Positive
error numbers are fatal, whilst negative numbers are warnings and
allow execution to continue.  On the second line is the function in
which the problem occurred and a brief textual explanation.  The
reason for sending warnings to standard output is so that they are
synchronised with any trace output.

If an error occurs a number of error messages may be produced on
standard error. Many of the functions in the \HTK\ Library do not exit
immediately when an error condition occurs, but instead print a message and return a failure
value back to their calling function. This process may be repeated several
times. When the \HTK\ Tool that called the function receives the failure
value, it exits the program with a fatal error message. Thus the
displayed output has a typical format as follows:

\begin{verbatim}
  ERROR [+nnnn] FunctionA: 'Brief explanation'
  ERROR [+nnnn] FunctionB: 'Brief explanation'
  ERROR [+nnnn] FunctionC: 'Brief explanation'
 FATAL ERROR - Terminating program HTool
\end{verbatim}

Error numbers in \HTK\ are allocated on a module by module
and tool by tool basis in blocks of 100 as shown by the table shown
overleaf.
Within each block of 100 numbers the first 20 (0 - 19)
and the final 10 (90-99) are reserved for standard types 
of error which are common to all tools and library 
modules.

All other codes are module or tool specific.

\mysect{Generic Errors}{generr}

\begin{itemize}
\erno{+??00} Initialisation failed \\
        The initialisation procedure for the tool produced an
        error. This could be due to errors in the command line
        arguments or configuration file.
\erno{+??01} Facility not implemented \\
        HTK does not support the operation requested.

\erno{+??05} Available memory exhausted \\
        The operation requires more memory than is available.
\erno{+??06} Audio not available \\
        The audio device is not available, either there is no driver for
        the current machine, the library was compiled with \texttt{NO\_AUDIO}
        set or another process has exclusive access to the audio device.

\begin{center}
\begin{tabular}{|c|c|c|c|}
\hline
HCopy    & 1000-1099     & HShell        & 5000-5099    \\
HList    & 1100-1199     & HMem          & 5100-5199    \\
HLEd     & 1200-1299     & HMath         & 5200-5299    \\
HLStats  & 1300-1399     & HSigP         & 5300-5399    \\
HDMan    & 1400-1499     &               &              \\
HSLab    & 1500-1599     & HAudio        & 6000-6099    \\
         &               & HVQ           & 6100-6199    \\
         &               & HWave         & 6200-6299    \\
HCompV   & 2000-2099     & HParm         & 6300-6399    \\
HInit    & 2100-2199     & HLabel        & 6500-6599    \\
HRest    & 2200-2299     &               &              \\
HERest   & 2300-2399     & HGraf         & 6800-6899    \\
HSmooth  & 2400-2499     &               &              \\
HQuant   & 2500-2599     & HModel        & 7000-7099    \\
HHEd     & 2600-2699     & HTrain        & 7100-7199    \\
         &               & HUtil         & 7200-7299    \\
         &               & HFB           & 7300-7399    \\
         &               & HAdapt        & 7400-7499    \\
HBuild   & 3000-3099     &               &              \\
HParse   & 3100-3199     & HDict         & 8000-8099    \\
HVite    & 3200-3299     & HLM           & 8100-8199    \\
HResults & 3300-3399     & HNet          & 8200-8299    \\
HSGen    & 3400-3499     & HRec          & 8500-8599    \\
HLRescore& 4000-4100     & HLat          & 8600-8699    \\
\hline
LCMap    & 15000-15099   & LAdapt        & 16400-16499  \\
LWMap    & 15100-15199   & LPlex         & 16600-16699  \\
LUtil    & 15200-15299   & HLMCopy       & 16900-16999  \\
LGBase   & 15300-15399   & Cluster       & 17000-17099  \\
LModel   & 15400-15499   & LLink         & 17100-17199  \\
LPCalc   & 15500-15599   & LNewMap       & 17200-17299  \\
LPMerge  & 15600-15699   &               &              \\
\hline
\end{tabular}
\end{center}

\erno{+??10} Cannot open file for reading \\
        Specified file could not be opened for reading.  The file may not
        exist or the filter through which it is read may not be set correctly.
\erno{+??11} Cannot open file for writing \\
        Specified file could not be opened for writing.  The directory may
        not exist or be writable by the user or the filter through which the
        file is written may not be set correctly.
\erno{+??13} Cannot read from file \\
        Cannot read data from file.  The file may have been truncated,  
        incorrectly formatted or the filter process may have died.
\erno{+??14} Cannot write to file \\
        Cannot write data to file.  The file system is full or the filter
        process has died.

\erno{+??15} Required function parameter not set \\
        You have called a library routine without setting one of the arguments.
\erno{+??16} Memory heap of incorrect type \\
        Some library routines require you to pass them a heap of a particular
        type.

\erno{+??19} Command line syntax error \\
        The command line is badly formed, refer to the manual or the command
        summary printed when the command is executed without arguments.

\erno{+??9?} Sanity check failed \\
        Several functions perform checks that structures are self consistent
        and that everything is functioning correctly.  When these sanity checks
        fail they indicate the code is not functioning as intended.  These
        errors should not occur and are not correctable by the user.
\end{itemize}

\mysect{Summary of Errors by Tool and Module}{errsum}

\begin{itemize}

\module{\htool{HCopy}}

\begin{itemize}
\erno{+1030}    Non-existent part of file specified\\
        \htool{HCopy} needed to access a non-existent part of the input file.
        Check that the times are specified correctly, that the label file
        contains enough labels and that it corresponds to the data file.

\erno{\pm 1031} Label file formatted incorrectly\\
        \htool{HCopy} is only able to properly copy label files with the
        same number of levels/alternatives.  When using labels with multiple
        alternatives only the first one is used to determine segment 
        boundaries.

\erno{+1032}    Appending files of different type/size/rate\\
        Files that are joined together must have the same parameter kind and 
        sample rate.

\erno{-1089}    ALIEN format set\\
        Input/output format has been set to \texttt{ALIEN}, ensure that 
        this was intended.

\end{itemize}

\module{\htool{HList}}

\module{\htool{HLEd}}

\begin{itemize}
\erno{+1230}    Edit script syntax error\\
        The \htool{HLEd} command script contains a syntax error, check the 
        input script against the descriptions of each command in 
        section~\ref{s:HLEd} or obtained by running \texttt{HLEd -Q}.

\erno{\pm 1231} Operation invalid\\
        You have either exceeded \htool{HLEd} limits on the number of 
        boundaries that can be specified, tried to perform an operation on 
        a non-existent level or tried to sort an auxiliary level into time 
        order.  None of these operations are supported.

\erno{+1232}    Cannot find pronunciation\\
        The dictionary does not contain a valid pronunciation (only occurs
        when attempting expansion from a dictionary).

\erno{-1289}    ALIEN format set\\
        Input/output format has been set to \texttt{ALIEN}, ensure that 
        this was intended.

\end{itemize}

\module{\htool{HLStats}}

\begin{itemize}
\erno{+1328}    Load/Make HMMSet failed\\
        The model set could not be loaded due to either an error opening the
        file or the data within being inconsistent.

\erno{\pm 1330} No operation specified\\
        You have invoked \htool{HLStats} but have not specified an operation 
        to be performed.

\erno{-1389}    ALIEN format set\\
        Input format has been set to \texttt{ALIEN}, ensure that this was 
        intended.

\end{itemize}

\module{\htool{HDMan}}

\begin{itemize}
\erno{\pm 1430} Limit exceeded\\
        \htool{HDMan} has several built in limits on the number of different
        pronunciation, phones, contexts and command arguments.  This error 
        occurs when you try to exceed one of them.  

\erno{\pm 1431} Item not found\\
        Could not find item for deletion.  Check that it actually occurs in the
        dictionary.

\erno{\pm 1450} Edit script file syntax error\\
        The \htool{HDMan} command script contains a syntax error, check the 
        input script against the descriptions of each command in 
        section~\ref{s:HDMan} or obtained by running \texttt{HDMan -Q}.

\erno{\pm 1451} Dictionary file syntax error\\
        One of the input dictionaries contained a syntax error.  Ensure that 
        it is in a \HTK\ readable form (see section~\ref{s:usehdman}).

\erno{\pm 1452} Word out of order in dictionary error\\
        Entries in the dictionary must be sorted into alphabetical (ASCII) order.

\end{itemize}

\module{\htool{HSLab}}

\begin{itemize}
\erno{-1589}    ALIEN format set\\
        Input/output format has been set to \texttt{ALIEN}, ensure that 
        this was intended.

\end{itemize}

\module{\htool{HCompV}}

\begin{itemize}
\erno{+2020}    HMM does not appear in HMMSet\\
        Supplied HMM filename does not appear in HMMSet. Check
        correspondence between HMM filename and HMMSet.

\erno{+2021}    Not enough data to calculate variance\\
        There are not enough frames of data to evaluate a
        reliable estimate of variance. Use more data.

\erno{+2028}    Load/Make HMMSet failed\\
        The model set could not be loaded due to either an error opening the
        file or the data within being inconsistent.

\erno{+2030}    Needs continuous models\\
        HCompV can only operate on models with an HMM set kind
        of \texttt{PLAINHS} or \texttt{SHAREDHS}.

\erno{+2039}    Speaker pattern matching failure\\
        The specified speaker pattern could not be matched against a
        given utterance file name.

\erno{+2050}    Data does not match HMM\\
        An aspect of the data does not match the equivalent aspect in 
        the HMMSet.  Check the parameter kind of the data.

\erno{-2089}    ALIEN format set\\
        Input format has been set to \texttt{ALIEN}, ensure that this was 
        intended.

\end{itemize}

\module{\htool{HInit}}

\begin{itemize}
\erno{+2120}    Unknown update flag\\
        Unknown flag set by \texttt{-u} option, use combinations of 
        \texttt{tmvw}.

\erno{+2121}    Too little data\\
        Not enough data to reliably estimate parameters.  Use more 
        training data.

\erno{+2122}    Segment with fewer frames than model states\\
        Segment may be too short to be matched to model, do not use
        this segment for training.

\erno{+2123}    Cannot mix covariance kind in a single mix\\
        Covariance kind of all mixture components in any one state must be
        the same.

\erno{+2124}    Bad covariance kind\\
        Covariance kind of mixture component must be either
        \texttt{FULLC} or \texttt{DIAGC}.

\erno{+2125}    No best mix found\\
        The Viterbi mixture component allocation failed to find a most likely
        component with this data.  Check that data is not corrupt and
        that parameter values produced by the initial uniform segmentation
        are reasonable.

\erno{+2126}    No path through segment\\
        The Viterbi segmentation failed to find a path through
        model with this data.  Check that data is not corrupt and that
        a valid path exists through the model.

\erno{+2127}    Zero occurrence count\\
        Parameter has had no data assigned to it and cannot be
        updated.  Ensure that each parameter can be estimated by
        using more training data or fewer parameters.

\erno{+2128}    Load/Make HMMSet failed\\
        The model set could not be loaded due to either an error opening the
        file or the data within being inconsistent.

\erno{+2129}    HMM not found\\
        HMM missing from HMMSet.  Check that the HMMSet is complete and 
        has not been corrupted.

\erno{+2150}    Data does not match HMM\\
        An aspect of the data does not match the equivalent aspect in 
        the HMMSet.  Check the parameter kind of the data.

\erno{+2170}    Index out of range\\
        Trying to access a mixture component or VQ index beyond the
        range in the current HMM.

\erno{-2189}    ALIEN format set\\
        Input format has been set to \texttt{ALIEN}, ensure that this was 
        intended.

\end{itemize}

\module{\htool{HRest}}

\begin{itemize}
\erno{+2220}    Unknown update flag\\
        Unknown flag set by \texttt{-u} option, use combinations of 
        \texttt{tmvw}.

\erno{+2221}    Too few training examples\\
        There are fewer training examples than the minimum set by the 
        \texttt{-m} option (default 3).  Either reduce the value specified
        by \texttt{-m} or use more training examples.

\erno{+2222}    Zero occurrence count\\
        Parameter has had no data assigned to it and cannot be
        updated.  Ensure that each parameter can be estimated by
        using more training data or fewer parameters.

\erno{+2223}    Floor too high\\
        Mix weight floor has been set so high that the sum over all 
        mixture components exceeds unity.  Reduce the floor value.

\erno{-2225}    Defunct Mix X.Y.Z \\
        Not enough training data to re-estimate the covariance vector of
        mixture component Z in stream Y of state X. The weight of the mixture
        component is set to 0.0 and it will never recover even with further
        training.

\erno{+2226}    No training data\\
        None of the supplied training data could be used to
        re-estimate the model. Data may be corrupt or has been floored.

\erno{+2228}    Load/Make HMMSet failed\\
        The model set could not be loaded due to either an error opening the
        file or the data within being inconsistent.

\erno{+2250}    Data does not match HMM\\
        An aspect of the data does not match the equivalent aspect in 
        the HMMSet.  Check the parameter kind of the data.

\erno{-2289}    ALIEN format set\\
        Input format has been set to \texttt{ALIEN}, ensure that this was 
        intended.

\end{itemize}

\module{\htool{HERest}}

\begin{itemize}
\erno{+2320}    Unknown update flag\\
        Unknown flag set by \texttt{-u} option, use combinations of 
        \texttt{tmvw}.

\erno{+2321}    Load/Make HMMSet failed\\
        The model set could not be loaded due to either an error opening the
        file or the data within being inconsistent.

\erno{-2326}    No transitions\\
        No transition out of an emitting state, ensure that
        there is a transition path from beginning to end of model.

\erno{+2327}    Floor too high\\
        Mix weight floor has been set so high that the sum over all 
        mixture components exceeds unity.  Reduce the floor value.

\erno{+2328}    No mixtures above floor\\
        None of the mixture component weights are greater than the floor value,
        reduce the floor value.

\erno{-2330}    Zero occurrence count\\
        Parameter has had no data assigned to it and cannot be
        updated.  Ensure that each parameter can be estimated by
        using more training data or fewer parameters.

\erno{-2331}    Not enough training examples\\
        Model was not updated as there were not enough training examples.
        Either reduce the minimum specified by \texttt{-m} or
        use more data.

\erno{-2389}    ALIEN format set\\
        Input format has been set to \texttt{ALIEN}, ensure that this was 
        intended.

\end{itemize}

\module{\htool{HSmooth}}

\begin{itemize}
\erno{+2420}    Unknown update flag\\
        Unknown flag set by \texttt{-u} option, use combinations of 
        \texttt{tmvw}.

\erno{+2421}    Invalid HMM set kind\\
        \htool{HSmooth} can only be used if HMM set kind is either
        \texttt{DISCRETE} or \texttt{TIED}.

\erno{+2422}    Too many monophones in list\\
        \htool{HSmooth} is limited to HMMSets containing fewer than
        500 monophones.

\erno{+2423}    Different number of states for smoothing\\
        Monophones and context-dependent models have differing
        numbers of states.

\erno{-2424}    No transitions\\
        No transition out of an emitting state, ensure that
        there is a transition path from beginning to end of model.

\erno{+2425}    Floor too high\\
        Mix weight floor has been set so high that the sum over all 
        mixture components exceeds unity.  Reduce the floor value.

\erno{-2427}    Zero occurrence count\\
        Parameter has had no data assigned to it and cannot be
        updated.  Ensure that each parameter can be estimated by
        using more training data or fewer parameters.

\erno{-2428}    Not enough training examples\\
        Model was not updated as there were not enough training examples.
        Either reduce the minimum specified by \texttt{-m} or
        use more data.

\erno{+2429}    Load/Make HMMSet failed\\
        The model set could not be loaded due to either an error opening the
        file or the data within being inconsistent.

\end{itemize}

\module{\htool{HQuant}}

\begin{itemize}
\erno{+2530}    Stream widths invalid\\
        The chosen stream widths are invalid.  Check that these match the 
        parameter kind and are specified correctly.

\erno{+2531}    Data does not match codebook\\
        Ensure that the parameter kind of the data matches that of the codebook
        being generated.

\end{itemize}

\module{\htool{HHEd}}

\begin{itemize}

\erno{+2628}    Load/Make HMMSet failed\\
        The model set could not be loaded due to either an error opening the
        file or the data within being inconsistent.

\erno{\pm 2630} Tying null or different sized items\\
        You have executed a tie command on items which do not have the 
        appropriate structure or the structures are not matched.  Ensure 
        that the item list refers only to the items that you wish to tie 
        together.

\erno{-2631}    Performing operation on no items\\
        The item list was empty, no operation is performed.

\erno{+2632}    Command parameter invalid\\
        The parameters to the command are invalid either because they
        refer to parts of the model that do not exist (for instance a
        state that does not appear in the model) or because they 
        do not represent an acceptable value (for instance HMMSet kind
        is not \texttt{PLAINHS}, \texttt{SHAREDHS}, \texttt{TIEDHS} or 
        \texttt{DISCRETEHS}).

\erno{+2634}    Join parameters invalid or not set\\
        Make sure than the join parameters (set by the \texttt{JO} command)
        are reasonable.  In particular take care that the floor is low enough 
        to ensure that when summed over all the mixture
        components the sum is below 1.0.

\erno{+2635}    Cannot find matching item\\
        Search for specified item was unsuccessful.  When this occurs with
        the \texttt{CL} or \texttt{MT} commands ensure that the appropriate 
        monophone/biphone models are in the current HMMSet.

\erno{-2637}    Small gConst\\
        A small gConst indicates a very low variance in that particular
        Gaussian.  This could be indicative of over-training of the models.

\erno{-2638}    No typical state\\
        When tying states together a search is performed for the distribution
        with largest variance and all tied states share this distribution.  If 
        this cannot be found the first in the list will be used instead.

\erno{-2639}    Long macro name\\
        In general macro names should not exceed 20 characters in length.

\erno{+2640}    Not implemented\\
        You have asked \htool{HHEd} to perform a function that is not 
        implemented.

\erno{+2641}    Invalid stream split\\
        The specified number/width of the streams does not agree with the 
        parameter kind/vector size of the models.

\erno{+2650}    Edit script syntax error\\
        The \htool{HHEd} command script contains a syntax error, check the 
        input script against the descriptions of each command in 
        section~\ref{s:HHEd} or obtained by running \texttt{HHEd -Q}.

\erno{+2651}    Command range error\\
        The value specified in the command script is out of range.  Ensure that
        the specified state exists and the the value given is valid.

\erno{\pm 2655} Stats file load error\\
        Either loading occupation statistics for the second time or executing 
        an operation that needs the statistics loaded without loading them.

\erno{+2660}    Trees file syntax error\\
        The trees file format did not correspond to that expected.  Ensure that
        the file is complete and has not been corrupted.

\erno{+2661}    Trees file macro/question not recognised\\
        The question or macro referred to does not exist.  Ensure that the file
        is complete and has not been corrupted.

\erno{+2662}    Trying to synthesize for unknown model\\
        There is no tree or prototype model for the new context.  Ensure that a
        tree has been constructed for the base phone.

\erno{\pm 2663} Invalid types to tree cluster\\
        Tree clustering will only work for single Gaussian diagonal
        covariance untied models of similar topology.

\end{itemize}

\module{\htool{HBuild}}

\begin{itemize}
\erno{\pm 3030} Mismatch between command line and language model\\
        Ensure that the \texttt{!ENTER} and \texttt{!EXIT} words are correctly 
        defined and that the supplied files are of the appropriate type.

\erno{\pm 3031} Unknown word\\
        Ensure that the word list corresponds to the language model/lattice
        supplied.

\end{itemize}

\module{\htool{HParse}}

\begin{itemize}
\erno{\pm 3130} Variable not defined\\
        You have referenced a network that has not yet been defined.  Check 
        that all networks are defined before they are referenced.

\erno{\pm 3131} Loop or word expansion error\\
        There is either a mismatch between the \texttt{WD\_BEGIN WD\_END} 
        pairs or a triphone loop is badly formed.

\erno{\pm 3132} Dictionary error\\
        When generating a dictionary a word exceeded the maximum number of 
        phones, a word occurred twice or no dictionary was produced.

\erno{\pm 3150} Syntax error in HParse file\\
        The \htool{HParse} network definition contains a syntax error, check 
        the input file against the network description in 
        section~\ref{s:HParse}.

\end{itemize}

\module{\htool{HVite}}

\begin{itemize}

\erno{+3228}    Load/Make HMMSet failed\\
        The model set could not be loaded due to either an error opening the
        file or the data within being inconsistent.

\erno{\pm 3230} Unsupported operation\\
        \htool{HVite} is not able to perform the operation requested

\erno{\pm 3231} Data does not match HMMs\\
        There is a mismatch between the data file and the HMMSet.  Ensure that
        the data is parameterised in the correct format and the configuration
        parameters match those used during training.

\erno{+3232} MMF Load Error\\
        The HMMSet does not contain a well-formed regression class tree.

\erno{+3233} Transcription empty\\
        In alignment mode a segment had an empty transcription and no
        boundary word was specified.

\erno{-3289}    ALIEN format set\\
        Input/output format has been set to \texttt{ALIEN}, ensure that 
        this was intended.

\end{itemize}

\module{\htool{HResults}}

\begin{itemize}
\erno{-3330}    Empty file\\
        The file was empty and will be skipped.

\erno{+3331}    Unknown label\\
        The label did not appear in the list supplied to HResults.
        This error will only occur if calculating confusion matrices so 
        normally the contents of the word list file will have no effect 
        on results.

\erno{+3332}    Too many labels\\
        \htool{HResults} will only generate confusion statistics for a small 
        number of labels.

\erno{\pm 3333} Cannot calculate word spot results\\
        When calculating word spotting results the label files need to have 
        both times and scores present.

\erno{-3389}    ALIEN format set\\
        Input format has been set to \texttt{ALIEN}, ensure that this was 
        intended.

\end{itemize}

\module{\htool{HSGen}}

\begin{itemize}
\erno{-3420}    Network malformed\\
        The word network is malformed. The information in a node (word
        and following arcs) is set incorrectly.

\end{itemize}

\module{\htool{HLRescore}}

\begin{itemize}

\erno{-4089}    ALIEN format set\\
        Input/output format has been set to \texttt{ALIEN}, ensure that 
        this was intended.

\end{itemize}

\module{\htool{HShell}}

\begin{itemize}
\erno{+5020}    Command line processing error
\erno{+5021}    Command line argument type error
\erno{+5022}    Command line argument range error\\
        The command line is badly formed.  Ensure that it matches the 
        syntax and values expected by the command (check the manual 
        page or the syntax obtained by running \htool{HTool} without any
        arguments).

\erno{+5050}    Configuration file format error\\
        \htool{HShell} was unable to parse the configuration. Check that
        it is of the format described in section~\ref{s:config}.

\erno{+5051}    Script file format error\\
        Check that the script file is just a list of file names and that
        if any file names are quoted that the quotes occur in pairs.

\erno{+5070}    Module version syntax error\\
        A module registered with HShell with an incorrectly formatted
        version string (which should be of the form
        \texttt{"!HVER!HModule: Vers.str [WHO DD/MM/YY]"}).

\erno{+5071}    Too many configuration parameters\\
        The size of the buffer used by one of the tools or modules to read
        its configuration parameters was exceeded.  Either reduce the total
        number of configuration parameters in the file or make more of then
        specific to their particular module rather than global.

\erno{+5072}    Configuration parameter of wrong type\\
        The configuration parameter is of the wrong type.  Check that its type 
        agrees with that shown in chapter~\ref{c:confvars}.

\erno{+5073}    Configuration parameter out of range\\
        The configuration parameter is out of range.  

\end{itemize}

\module{\htool{HMem}}

\begin{itemize}
\erno{+5170}    Heap parameters invalid\\
        You have tried to create a heap with unreasonable parameters.  Adjust 
        these so that the growth factor is positive and the initial block 
        size is no larger than the maximum.  For \texttt{MSTAK} the element 
        size should be 1.

\erno{+5171}    Heap not found\\
        The specified heap could not be found, ensure that it has not been 
        deleted or memory overwritten.

\erno{+5172}    Heap does not support operation\\
        The heap is of the wrong type to support the requested operation.  In 
        particular it is not possible to \texttt{Reset} or \texttt{Delete} a
        \texttt{CHEAP}.

\erno{+5173}    Wrong element size for MHEAP\\
        You have tried to allocate an item of the wrong size from a 
        \texttt{MHEAP}. All items on a \texttt{MHEAP} must be of the same size.

\erno{+5174}    Heap not initialised\\
        You have tried to allocate an item on a heap that has not yet been 
        created.  Ensure that \texttt{CreateHeap} is called to initialise 
        the heap before any items are allocated from it.

\erno{+5175}    Freeing unseen item\\
        You have tried to free an item from the wrong heap.  This can occur
        if the wrong heap is specified, the item pointer has been corrupted 
        or the item has already been freed implicitly by a 
        \texttt{Reset/DeleteHeap} call.  

\end{itemize}
 
\module{\htool{HMath}}
 
\begin{itemize}
\erno{+5220}    Singular covariance matrix\\
        The covariance matrix was not invertible.  This may indicate a lack
        of training data or linearly dependent parameters.

\erno{+5270}    Size mismatch\\
        The input parameters were of incompatible sizes.

\erno{+5271}    Log of negative\\
        Result would be logarithm of a negative number.

\end{itemize}

\module{\htool{HSigP}}

\begin{itemize}
\erno{+5320}    No results for WaveToLPC\\
        Call did not include Vectors for the results.

\erno{+5321}    Vector size mismatch\\
        Input vectors were of mismatched sizes.

\erno{-5322}    Clamped samples during zero mean\\
        During a zero mean operation samples were clipped as they were outside
        the allowable range.

\end{itemize}

\module{\htool{HAudio}}

\begin{itemize}
\erno{+6020}    Replay buffer not active\\
        Attempt to access a replay buffer when one was not attached.

\erno{+6021}    Cannot StartAudio without measuring silence\\
        An attempt was made to start audio input through the silence detector 
        without first measuring or supplying the background silence values.

\erno{+6070}    Audio frame size/rate invalid\\
        The choice of frame period and window duration are invalid.  Check
        both these and the sample rate.

\erno{-6071}    Setting speech threshold below silence\\
        The thresholds used in the speech detector have been set so that the
        threshold for detecting speech is set below that of detecting silence.

\end{itemize}

\module{\htool{HVQ}}

\begin{itemize}
\erno{+6150}    VQ file format error\\
        The VQ file was incorrectly formatted.  Ensure that the file is 
        complete and has not been corrupted.

\erno{+6151}    VQ file range error\\
        A value from the VQ file was out of range.  Ensure that the file is 
        complete and has not been corrupted.

\erno{+6170}    Magic number mismatch\\
        The VQ magic number (normally based on parameter kind) does not match
        that expected.  Check that the parameter kind used to quantise the data
        and create the VQ table matches the current parameter kind.

\erno{+6171}    VQ table already exists\\
        All VQ tables must have distinct names.  This error will occur if you
        try to create or load a VQ table with the same name as one already
        loaded.

\erno{+6172}    Invalid covariance kind\\
        Entries in VQ tables must have either \texttt{NULLC}, \texttt{FULLC} or
        \texttt{INVDIAGC} covariance kind.

\erno{+6173}    Node not in table\\
        A node was missing from the VQ table.  Ensure that the VQ table was 
        properly created or that the file was complete.

\erno{+6174}    Stream codebook mismatch\\
        The number or size of streams in the VQ table does not match that 
        requested.

\end{itemize}

\module{\htool{HWave}}

\begin{itemize}
\erno{+6220}    Cannot fseek/ftell\\
        Unless the wave file is read through a pipe fseek and ftell are 
        expected to work correctly so that \htool{HWave} can calculate the 
        file size.  If this error occurs when using an input pipe, supply 
        the number of samples in the file using the configuration variable
        \texttt{NSAMPLES}.

\erno{+6221}    File appears to be a infinite\\
        \htool{HWave} cannot determine the size of the file.

\erno{+6230}    Config parameter not set\\
        A necessary configuration parameter has not been set.  Determine the 
        correct value and place this in the configuration file before 
        re-invoking the tool.

\erno{+6250}    Premature end of header\\
        \htool{HWave} could not read the complete file header.

\erno{+6251}    Header contains invalid data\\
        \htool{HWave} was unable to successfully parse the header.  The header
        is invalid, of the wrong type or be a variation that \htool{HWave} does
        not handle.

\erno{+6252}    Header missing essential data\\
        The header was missing a piece of information necessary for 
        \htool{HWave} to load the file.  Check the processing of the 
        input file and re-process if necessary.

\erno{+6253}    Premature end of data\\
        The file ended before all the data was read correctly.  Check that the
        file is complete, has not been corrupted and where necessary 
        \texttt{NSAMPLES} is set correctly.

\erno{+6254}    Data formated incorrectly\\
        The data could not be decoded properly.  Check that the file was 
        complete and processed correctly.

\erno{+6270}    File format invalid\\
        The file format is not valid for the operation requested.

\erno{+6271}    Attempt to read outside file\\
        You have tried to read a sample outside of the waveform file.

\end{itemize}

\module{\htool{HParm}}

\begin{itemize}
\erno{+6320}    Configuration mismatch\\
        The data file does not match the configuration.  Check the 
        configuration file is correct.

\erno{+6321}    Invalid parameter kind\\
        Parameter kind is not valid.  Check the configuration file.

\erno{+6322}    Conversion not possible\\
        The specified conversion is not possible.  Check the configuration is
        correct and re-code the data from waveform files if necessary.

\erno{+6323}    Audio error\\
        An audio error has been detected.  Check the \htool{HAudio} 
        configuration and the audio device.

\erno{+6324}    Buffer not initialised\\
        Ensure that the buffer is used in the correct manner.

\erno{+6325}    Silence detection failed\\
        The silence detector was not initialised correctly before use.

\erno{+6328}    Load/Make HMMSet failed\\
        The model set could not be loaded due to either an error opening the
        file or the data within being inconsistent.

\erno{+6350}    CRC error\\
        The CRC does not match that of the data.  Check the data file is 
        complete and has not been corrupted.

\erno{-6351}    Byte swapping not possible\\
        \htool{HParm} will attempt to byte swap parameter files but this 
        may not work if the floating point representation of the machine 
        that generated the file is different from that which is reading it.

\erno{+6352}    File too short to parameterise\\
        The file does not contain enough data to produce a single observation.
        Check the file is complete and not corrupt.  If it is, it should be 
        discarded.

\erno{+6370}    Unknown parameter kind\\
        The specified parameter kind is not recognised.  Refer to 
        section~\ref{s:spiosum} for a list of allowable parameter kinds
        and qualifiers.

\erno{+6371}    Invalid parameters for coding\\
        The chosen parameters are not valid for coding.  Choose different ones.

\erno{+6372}    Stream widths not valid\\
        Cannot split the data into the specified number of streams.  Check that
        the parameter kind is correct and matches any models used.

\erno{+6373}    Buffer/observation mismatch\\
        The observation parameter kind should match that of the input buffer.
        Check that the configuration parameter kind is correct and matches 
        that of any models used.

\erno{+6374}    Buffer size too small for window\\
        Calculation of delta parameters requires a window larger than the
        buffer size chosen.  Increase the size of the buffer.

\erno{+6375}    Frame not in buffer\\
        An attempt was made to access a frame that does not appear in the 
        buffer.  Make sure that the file actually contains the specified frame.

\erno{+6376}    Mean/Variance normalisation failed\\
        The mean or variance normalisation vector from the file
        specified by the normalisation dir and mask cannot be applied.
        Make sure the file format is correct and the vectors are of
        the right dimension.

\end{itemize}

\module{\htool{HLabel}}

\begin{itemize}
\erno{+6520}    MLF index out of range\\
        An attempt was made to access an MLF that has not been loaded or to 
        load too many MLFs.

\erno{+6521}    fseek/ftell not possible\\
        \htool{HLabel} needs random access to MLFs.  This error is generated
        when this is not possible (for instance if access is via a pipe).

\erno{+6550}    HTK format error
\erno{+6551}    MLF format error
\erno{+6552}    TIMIT format error
\erno{\pm 6553} ESPS format error
\erno{+6554}    SCRIBE format error\\
        A label file was formatted incorrectly.  Label
        file formats are described in chapter~\ref{c:labels}.

\erno{+6570}    Level out of range\\
        Attempted to access a non-existent label level.  Check that the correct
        label file has been loaded.

\erno{+6571}    Label out of range\\
        Attempted to access a non-existent label.  Check that the correct
        label file has been loaded and that the correct level is being 
        accessed.

\erno{+6572}    Invalid format\\
        The specified file format is not valid for the particular operation.

\end{itemize}

\module{\htool{HModel}}

\begin{itemize}
\erno{+7020}    Cannot find physical HMM\\
        No physical HMM exists for a particular logical model.  Check that the
        HMMSet was loaded or created correctly.

\erno{+7021}    INVDIAG internal format\\
        Attempts to load or save models with \texttt{INVDIAG} covariance kind
        will fail as this is a purely internal model format.

\erno{\pm 7023} varFloor should be variance floor\\
        \htool{HModel} reserves the macro name \texttt{varFloorN} as the 
        variance floor for stream \texttt{N}.  These should be variance 
        macros (type \texttt{v}) of the correct size for the particular stream.

\erno{+7024}    Variance tending to 0.0\\
        A variance has become too low.  Start using a variance floor or 
        increase the amount of training data.
        
\erno{+7025} Bad covariance kind\\
        The particular functionality does not support the covariance
        kind of the mixture component.

\erno{+7030}    HMM set incomplete or inconsistent\\
        The HMMSet contained missing or inconsistent data.  Check that the 
        file is complete and has not been corrupted.

\erno{+7031}    HMM parameters inconsistent\\
        Some model parameters were inconsistent.  Check that the file is 
        complete and has not been corrupted.

\erno{\pm 7032} Option mismatch\\
        All HMMs in a particular set must have consistent options.

\erno{+7035}    Unknown macro\\
        Macro does not exist.  Check that the name is correct and appears 
        in the HMMSet.

\erno{+7036}    Duplicate macro\\
        Attempted to create a macro with the same name as one already present.
        Choose a different name.

\erno{+7037}    Invalid macro\\
        Macro had invalid type.  See section~\ref{s:HMMmac} describes the 
        allowable macro types.

\erno{+7050}    Model file format error
\erno{+7060}    HMM List format error\\
        The file was formated incorrectly.  Check the file is complete and
        has not been corrupted.

\erno{+7070}    Invalid HMM kind\\
        Invalid HMMSet kind.  Check that this is specified correctly.

\erno{+7071}    Observation not compatible with HMMSet\\
        Attempted to calculate an observation likelihood for an observation
        not compatible with the HMMSet.  Check that the parameter kind is
        set correctly.

\end{itemize}

\module{\htool{HTrain}}

\begin{itemize}
\erno{+7120}    Clustering failed\\
        Almost certainly due to a lack of data, reduce the
        number of clusters requested  or increase amount of data.

\erno{+7150}    Accumulator file format error\\
        Cannot read an item from an accumulator file. Check
        that file is complete and not corrupted.

\erno{+7170}    Unsupported covariance kind\\
        Covariance kind must be either \texttt{FULLC}, \texttt{DIAGC} or 
        \texttt{INVDIAGC}.

\erno{+7171}    Item out of range\\
        Attempt made to access data beyond expected range. Check that 
        the item number is correct.

\erno{+7172}    Tree size must be power of 2\\
        Requested codebook size must be a power of 2 when
        using tree based clustering.

\erno{-7173}    Segment empty\\
        Empty data segment in file. Check that file has not
        become corrupted and that the start and end segment times
        are correct.

\end{itemize}

\module{\htool{HUtil}}

\begin{itemize}
\erno{+7220}    HMMSet empty\\
        A scan was initiated for a HMMSet with no members.

\erno{+7230}    Item list parse error\\
        The item list syntax was incorrect.  Check the item list specification
        in section~\ref{s:HHEd}.

\erno{+7231}    Item list type error\\
        Each item in a particular list should be of the same type and size.

\erno{+7250}    Stats file format error\\
        Stats file is of wrong format.  Note the format of the stats file has 
        changed in HTK\_V2.0 and old files will need converting to the new
        format.

\erno{+7251}    Stats file model error\\
        A model name encountered in the stats file is invalid check that the
        model set corresponds to that used to generate the stats file and that
        the stats file is complete and has not been corrupted.

\erno{+7270}    Accessing non-existent macro\\
        Attempt to perform operation on non-existent macro.

\erno{+7271}    Member id out of range\\
        Attempt to perform set operation on out of range member.

\end{itemize}

\module{\htool{HFB}}

\begin{itemize}

\erno{+7321}    Unknown model\\
        Model in HMM List not found in HMMSet, check that the
        correct HMM List is being used.

\erno{+7322}    Invalid output probability\\
        Mixture component probability has not been set.  This should
        not occur in normal use.

\erno{+7323}    Beta prune failed on taper\\
        Utterance is possibly too short for minimum duration
        of model sequence. Check transcription.

\erno{-7324}    No path through utterance\\
        No path was found on the beta training pass, relax the
        pruning threshold.

\erno{-7325}    Empty label file\\
        No labels found in label file, check label file.

\erno{+7326}    Single-pass retraining data mismatch\\
        Paired training files must contain the same number of observations.  
        Use original data to re-parameterise.

\erno{\pm 7332}  HMM with unreachable states\\
        HMM has an unreachable state, check transition matrix.\\

\erno{-7333} Transition matrix with discontinuity\\
        Check transition matrix.\\        

\erno{+7350}    Data does not match HMM\\
        An aspect of the data does not match the equivalent aspect in 
        the HMMSet.  Check the parameter kind of the data.

\end{itemize}

\module{\htool{HDict}}

\begin{itemize}
\erno{+8050}    Dictionary file format error\\
        The dictionary file is not correctly formatted.  
        Section~\ref{s:usehdman} describes the HTK dictionary file format.

\end{itemize}

\module{\htool{HLM}}

\begin{itemize}
\erno{+8150}    LM syntax error\\
        The language model file was formatted incorrectly.  Check the file is
        complete and has not been corrupted.

\erno{\pm 8151} LM range error\\
        The specified value(s) for the language model probability are not 
        valid.  Check the input files are correct.

\end{itemize}

\module{\htool{HNet}}

\begin{itemize}
\erno{+8220}    No such word\\
        The specified word does not exist or does not have a valid 
        pronunciation.

\erno{-8221}    Duplicate pronunciations removed\\
        During network generations duplicate identical pronunciations
        of the same word are removed.

\erno{+8230}    Contexts not consistent\\
        \htool{HNet} can only deal with the standard HTK method for specifying
        context \texttt{left-phone+right} and will only allow context free 
        phones if they are context independent and only form part of the word.
        This may be indicative of an inconsistency between the symbols in the dictionary
        and the hmms as defined. There may be a model/phone in the dictionary that has 
        not been defined in the HMM list or may not have a corresponding model.
        See also section~\ref{s:netexpand}  on context expansion.


\erno{+8231}    No such model\\
        A particular model could not be found.  Make sure that the network is
        being expanded in the correct fashion and then ensure that your HMM
        list will cover all required contexts.

\erno{+8232}    Lattice badly formed\\
        Could not convert lattice to network.  The lattice should have a single
        well defined start and a single well defined end.  When cross word 
        expansion is being performed the number of \texttt{!NULL} words that 
        can be concatenated in a string is limited.

\erno{+8250}    Lattice format error\\
        The lattice file is formatted incorrectly.  Ensure that the lattice
        is of the format described in chapter~\ref{c:htkslf}.

\erno{+8251}    Lattice file data error\\
        The value specified in the lattice file is invalid.

\erno{+8252}    Lattice file with multiple start/end nodes\\
        A lattice should have only one well defined start node and one
        well defined end node.

\erno{+8253}    Lattice with invalid sub lattices\\
        The sub lattices referred to by the main lattices are
        malformed.

\end{itemize}


\module{\htool{HRec}}

\begin{itemize}
\erno{\pm 8520} Invalid HMM\\
        One of the HMMs in the network is invalid.  Check that the HMMSet
        has been correctly initialised.

\erno{+8521}    Network structure invalid\\
        The network is incorrectly structured.  Take care to avoid loops
        that can be traversed without consuming observations (this may occur 
        if you introduce any 'tee' words in which all the models making up that
        word contain tee-transitions).  Also ensure that the recogniser and
        the network have been created and initialised correctly.

\erno{+8522}    Lattice structure invalid\\
        The lattice was incorrectly formed.  Ensure that the lattice was
        created properly.

\erno{\pm 8570} Recogniser not initialised correctly\\
        Ensure the recogniser is initialised and used correctly.

\erno{+8571}    Data does not match HMMs\\
        The observation does not match the HMM structure.  Check the parameter
        kind of the data and ensure that the data is matched to the HMMs.

\end{itemize}


\module{\htool{HLat}}

\begin{itemize}

\erno{8621}    Lattice incompatible with dictionary\\
        The lattice refers to a pronunciation variant (filed
        \texttt{v=}) that doesn't exist in the current dictionary.

\erno{\pm 8622}    Lattice structure invalid\\
        The lattice does not meet the requirements for some operation.
        All lattices must have unique start and end nodes and for many
        operations the lattices need to be acyclic (i.e.\ be a
        Directed Acyclic Graph).

\erno{8623}    Start or end word not found\\
        The specified lattice start or end word could not be found in
        the dictionary.

\erno{8624}    Lattice end node label invalid\\
        The lattice end node must either be labelled with \verb|!NULL|
        or the specified end word (default: \verb|!SENT_END|)

\erno{-8630}    LLF file not found\\
        The specified LLF file could not be found or isn't in the
        right format.

\erno{8631}    Lattice not found in LLF file\\
        A lattice couldn't be found in the LLF file. Note that the
        order in the LLF file must correspond to the order of processing.

\erno{8632}    Lattice not found\\
        The specified lattice file could not be opened.

\erno{8690}    Lattice operation not supported\\
        The requested operation is not supported, yet.
\erno{8691}    Lattice processing sanity check failed\\
        During processing an internal sanity check failed. This should
        never happen..

\end{itemize}


\module{\htool{HGraf}}

\begin{itemize}
\erno{+6870}    X11 error\\
        Ensure that the \texttt{DISPLAY} variable is set and that the
        X11 window system is configured correctly.

\end{itemize}


% Language modelling libraries and tools

% LCMap
\module{\htool{LCMap}}

\begin{itemize}
\erno{+15050}  Unlikely num map entries[\texttt{n}] in \texttt{XYZ}\\
        A negative or infeasibly large number of class map entries
        have been specified.

\erno{+15051} ReadMapHeader: UNKxxx configs must be set for hdrless map\\
        There is no header on the map so you must set UNKNOWNID and UNKNOWNNAME.

\erno{+15052} No name in \texttt{XYZ}\\
        No NAME header in class map.

\erno{+15053} Unknown escmode \texttt{XYZ} in \texttt{XYZ}\\
        ESCMODE header must specify either HTK or RAW.

\erno{+15054} Class name  \texttt{XYZ} duplicate in \texttt{XYZ}\\
        Two classes in the class map have the same name, which is not allowed.

\erno{+15055} Bad index \texttt{n} for class \texttt{XYZ} in \texttt{XYZ}\\
        A class index less than 1 or greater than or equal to BASEWORDNDX (defined
        at compile time in \htool{LWMap} - default is 65536) was found
        in the class map.  If you need more than \htool{BASEWORDNDX}
        classes then you must recompile \HTK\ with a new base word
        value.

\erno{+15056} Number of entries =  \texttt{n} for class \texttt{XYZ} in \texttt{XYZ}\\
        There must be at least one member in each class - empty
        classes are not allowed.

\erno{+15057} Bad type \texttt{XYZ} for class \texttt{XYZ} in \texttt{XYZ}\\
        Classes must be defined using either IN or NOTIN.

\erno{+15058} A class is in its own exclusive list. This typically happens when a class map 
              is specified as a plain list of words. Such list is by default assumed to be 
              a list of words excluded from class !!UNK. The error is triggered when !!UNK
              is in the word list. !!UNK must be removed from the list.

\end{itemize}


% LWMap
\module{\htool{LWMap}}

\begin{itemize}
\erno{+15150}   Word list/word map file format error\\
        Check that the word list/word map file is correctly formatted.

\erno{+15151}  Unlikely num map entries[\texttt{n}] in \texttt{XYZ}\\
        A negative or infeasibly large number of word map entries
        have been specified.

\erno{+15152} No NAME header in \texttt{XYZ}\\
        No NAME header in word map.

\erno{+15153} No SEQNO header in \texttt{XYZ}\\
        No SEQNO header in word map.

\erno{+15154} Unknown escmode \texttt{XYZ} in \texttt{XYZ}\\
        ESCMODE header must specify either HTK or RAW.

\erno{+15155} Word name \texttt{XYZ} is duplicated in \texttt{XYZ}\\
        There are duplicate words in the word map, which is not allowed.

\end{itemize}


% LUtil
\module{\htool{LUtil}}

\begin{itemize}
\erno{+15250}   Header format error\\
        Ensure that word maps and/or n-gram files used by the program start
        with the appropriate header.
\end{itemize}


% LGBase
\module{\htool{LGBase}}

\begin{itemize}
\erno{+15330}   n-gram file consistency check failure\\
        The n-gram file is incompatible with other resources used by the
        program.

\erno{+15340}   File \texttt{XYZ} is \texttt{n}-gram but inset is \texttt{n}-gram\\
        The specified input gram file is not of the expected gram size.

\erno{+15341}   Requested N[\texttt{n}] greater than gram size [\texttt{n}]\\
        An n-gram was requested which was larger than any of those
        supplied in the input files.

\erno{+15345}   n-grams out of order\\
        The input gram file is not correctly sorted.

\erno{+15350}   n-gram file format error\\
        Ensure that n-gram files used by the program are formatted correctly
        and start with the appropriate header.
\end{itemize}


% LModel
\module{\htool{LModel}}

\begin{itemize}
\erno{+15420}   Cannot find n-gram component\\
        The internal structure of the language model is corrupted. This error
        is usually caused when an n-gram $(a,b,c)$ is encountered without
        the presence of n-gram $(a,b)$.
\erno{+15430}   Incompatible probability kind in conversion\\      
        The currently used language model does not allow the required
        conversion operation. This error is caused by attempting to prune a
        model stored in the ultra file format.
\erno{+15440}   Cannot prune models in ultra format\\
        Pruning of language models stored in {\em ultra} file format is not
        supported.
\erno{+15445}   Word ID size error\\
        Language models with vocabularies of over 65,536 words require the
        use of larger word identifiers. This is a sanity check error.
\erno{-15450}   Word \texttt{XYZ} not in unigrams - skipping n-gram.\\
        There should be a unigram count for each word in other length grams.
\erno{+15450}   Language model file format error\\
        The language model file is formatted incorrectly. Check the file is
        complete and has not been corrupted.
\erno{-15451}   Extraneous line warning\\
        Extra lines were found on the end of a file and are being ignored.
\erno{-15460}   Model order reduced\\
        Due to the effects of pruning the model order is automatically reduced.
\end{itemize}


% LPCalc
\module{\htool{LPCalc}}

\begin{itemize}
\erno{+15520}   Unable to find FLEntry to attach\\
        Indicates that the LM data structures are corrupt. This is normally caused
        by NGram files which have not been sorted.
\erno{+15525}   Attempt to overwrite entries when attaching\\
        Indicates that the LM structure is corrupt. Ensure that the word map file
        used is suitable for decoding the NGram database files.
\erno{-15540}   \texttt{n}-gram cutoff out of range\\
        An inapplicable cutoff was ignored.
\erno{+15540}   Pruning error\\
        The pruning parameters specified are not compatible with the parameters 
        of the language model.
\end{itemize}


% LPMerge
\module{\htool{LPMerge}}

\begin{itemize}
\erno{+15620}   Unable to find word in any model\\
        Indicates that the target model vocabulary contains a word which cannot
        be found in any of the source models.
\end{itemize}



% LPlex
\module{\htool{LPlex}}

\begin{itemize}
\erno{+16620} symbol {\tt XYZ} not in word list\\
        The sentence start symbol, sentence end symbol and OOV symbol (only if
        OOVs are to be included in the perplexity calculation) must be in the
        language model's vocabulary. Note that the vocabulary list is either 
        specified with the {\tt -w} option or is implicitly derived from the 
        language model.

\erno{+16625} Unable to find word {\tt XYZ} in any model\\
        Ensure that all words in the vocabulary list specified with the {\tt -w}
        option are present in at least one of the language models.

\erno{+16630} Maximum number of unique OOVs reached\\
        Too many OOVs encountered in the input text.

\erno{-16635} Transcription file {\tt fn} is empty\\
        The label file does not contain any words.

\erno{-16640} Word too long, will be split: {\tt XYZ}\\
        The word read from the input stream is of over 200 characters.

\erno{-16645} Text buffer size exceeded ({\tt n})\\ 
        The maximum number of words allowed in a single utterance has been 
        reached.

\erno{+16650} Maximum utterance length in a label file exceeded (limit
        is compiled to be {\tt n} tokens)\\
        No label file utterance end has been encountered within
        {\tt n} tokens -- perhaps this is a text file and you forgot
        to pass the {\tt -t} option?
\end{itemize}


% HLMCopy
\module{\htool{HLMCopy}}

\begin{itemize}
\erno{+16920} Maximum number of phones reached\\
        When \htool{HLMCopy} is used to copy dictionaries, the target dictionary's
        phone table is composed by combining the phone tables of all source
        dictionaries. Check that the number of different phones resulting from
        combining the phone tables of the source dictionaries does not exceed the
        internally set limit.

\erno{+16930} Cannot find definition for word {\tt XYZ}\\
        When copying dictionaries, ensure that each word in the vocabulary list
        occurs in at least one source dictionary.
\end{itemize}


% Cluster
\module{\htool{Cluster}}

\begin{itemize}
\erno{+17050} Word {\tt XYZ} found in class map but not in word map\\
        All words in the class map must be found in the word map too.

\erno{-17051} Unknown word token {\tt XYZ} was explicitly given with -u, but
          does not occur in the word map\\
        This warning appears if you specify an unknown word token
          which is not found in the word map.

\erno{+17051} Token not found in word list\\
        Sentence start, end and unknown (if used) tokens must be found
        in the word map.

\erno{+17052} Not all words were assigned to classes\\
        A classmap was imported which did not include all words in the
        word map.

\erno{-17053} Word {\tt XYZ} is in word map but not in any gram files\\
        The stated word will remain in whichever class it is already
        in - either as defaulted to or supplied via the input class map.
\end{itemize}


\end{itemize}


%%% Local Variables: 
%%% mode: latex
%%% TeX-master: "htkbook"
%%% End: 
